%!TEX root = /Users/olav/Desktop/Doktorarbeit/Causality/challenge/documentation/CHA-readme.tex

\begin{tikzpicture}[node distance = 3cm, auto]
  % Place nodes
  \node [block] (topos) {Generate topologies};
  \node [block, right of=topos] (spikes) {Simulate spiking};
  \node [block, right of=spikes] (fluoro) {Simulate calcium fluorescence};
  \node [block, right of=fluoro] (rec) {Compute effective connectivity};
  \node [block, right of=rec] (rocs) {Compute ROC curves};
  % \node [block, below of=identify] (evaluate) {evaluate candidate models};
  % \node [block, left of=evaluate, node distance=3cm] (update) {update model};
  % \node [decision, below of=evaluate] (decide) {is best candidate better?};
  % \node [block, below of=decide, node distance=3cm] (stop) {stop};
  % Draw edges
  \path [line] (topos) -- (spikes);
  \path [line] (spikes) -- (fluoro);
  \path [line] (fluoro) -- (rec);
  \path [line] (rec) -- (rocs);
  % \path [line] (identify) -- (evaluate);
  % \path [line] (evaluate) -- (decide);
  % \path [line] (decide) -| node [near start] {yes} (update);
  % \path [line] (update) |- (identify);
  % \path [line] (decide) -- node {no}(stop);
  % \path [line,dashed] (expert) -- (init);
  % \path [line,dashed] (system) -- (init);
  % \path [line,dashed] (system) |- (evaluate);
\end{tikzpicture}
